\subsection{Windows10でのインストール}
Windows10上Windows Subsystem for Linux (WSL, 仮想的なLinux環境のようなもの) を構築し, そのlinux環境の上でROOTをインストール, 実行する.2022年3月現在,WSLには2つのバージョンがあり, それぞれWSL, WSL2と呼称されている.ここではWSL2ではなくWSLを用いたROOTのインストール方法を紹介する.

\subsubsection{WSLの導入}
まず, Windows 側で Linux Subsystem を有効化します.
\begin{enumerate}
    \item スタートボタンを右クリックし, 「アプリと機能」をクリックします.
    \item 右上の「プログラムと機能」をクリックします.
    \item 左サイドバーの「Windows の機能の有効化または無効化」をクリックします.
    \item 「Windows Subsystem for Linux」を探して, チェックを入れます.
    \item 「OK」を押すと, インストールが始まります.インストール後, 再起動します.
\end{enumerate}

\subsubsection{Ubuntuのインストール}
\begin{enumerate}
    \item Microsoft Store で「WSL」と検索します.もしくは「Ubuntu」.
    \item アプリの中から「Ubuntu 20.04」を探し, インストールします.
    \item インストール作業が終わったら, スタートに追加される「Ubuntu 20.04」をを起動します.
    \item ユーザー名とパスワード を要求されるため, プロンプトに従って入力していきます.
\end{enumerate}
\begin{lstlisting}[caption=表示されるスクリプト例]
Installing, this may take a few minutes...
Please create a default UNIX user account. The username does not need to match your Windows username.
For more information visit: https://aka.ms/wslusers
Enter new UNIX username: [user name]
Enter new UNIX password: [password]
Retype new UNIX password: [password]
passwd: password updated succsessfully
Installation successful!
To run a command as administrator (user ``root''), use ``sudo''.
See ``man sudo_root'' for details.
\end{lstlisting}
\begin{enumerate}
    \setcounter{enumi}{4}
    \item パッケージのアップデート\\
        パッケージのアップデートは日々行ったほうが良いです.
\end{enumerate}
\begin{lstlisting}
$ sudo apt update
$ sudo apt upgrade
\end{lstlisting}

\subsubsection{X windowのインストール}
\begin{enumerate}
    \item \url{https://sourceforge.net/projects/vcxsrv/} にアクセスし, 「Download」をクリックします.
    \item インストーラを起動し, オプションやインストール先などを指定してインストールします.
        \begin{itemize}
            \item Installation Option : 全てにチェックする. (Full)
            \item Installation Folder : C:¥Program File¥VcXsrv
        \end{itemize}
    \item インストール作業が終わったら, スタートに追加される「VcXsrv」をを起動します.
    \item 起動するとウィザード画面が表示される.それぞれ選択し, 設定を完了させます.
        \begin{itemize}
            \item Display settings : 好きなもので
            \item Client startup : 次へ (start no client)
            \item Extra settings : 次へ (Clipboard - Primary Selection, Native opengl
            \item Save configuration : 「Save configuration」をクリック
        \end{itemize}
    \item システムトレイにVcXsrvのアイコンが表示されていれば OK です.
    \item WSLにX window serverをインストールする
        \begin{itemize}
            \item Ubuntuを起動します
            \item 以下を実行し, インストールを開始します
                \begin{lstlisting}
$ sudo apt install xfce4-terminal xfce4 x11-apps
                \end{lstlisting}
            \item \textasciitilde/bashrcに以下を追記します
                \begin{lstlisting}
export DISPLAY=:0.0
                \end{lstlisting}
            \item 以下を実行します
                \begin{lstlisting}
$ source ~/.bashrc
                \end{lstlisting}
        \end{itemize}
    \item 動作確認
        \begin{itemize}
            \item 以下を実行し, 目玉が表示されたらインストールが正常に完了しています
                \begin{lstlisting}
$ xeyes
                \end{lstlisting}
        \end{itemize}
\end{enumerate}

\subsubsection{ROOTのビルド}
この項は\url{}を参考に書かれています.\\
ROOTをUbuntu上で動かすにはあらかじめいくつかのパッケージをインストールしておく必要があります.次のURLにあるUbuntuの項のパッケージをインストールしてください.\\
\url{https://root.cern/install/dependencies/}\\
例としては以下の通りです.
\begin{lstlisting}
$ sudo apt-get install dpkg-dev cmake g++ gcc binutils libx11-dev libxpm-dev libxft-dev libxext-dev python libssl-dev
\end{lstlisting}
依存関係のパッケージが揃ったのでビルドを進めます.ビルドの方法は\url{https://root.cern/install/build_from_source/#quick-start}に全て書いてあります.例を以下に示します.
\begin{lstlisting}
$ cd ~
$ mkdir root_build
$ git clone --branch latest-stable https://github.com/root-project/root.git root_src
$ cd root_build
$ sudo cmake -DCMAKE_INSTALL_PREFIX=/opt/root ../root_src
$ sudo cmake --build . --target install -j4
\end{lstlisting}
ビルドが成功したら, パスを通して終了です.
\$HOME/.bashrcに以下を追記し, シェルを再起動します.
\begin{lstlisting}
$ source /opt/root/bin/thisroot.sh
\end{lstlisting}


\clearpage
\subsection{MacOSでのインストール}
\subsubsection{Homebrewのインストール}
Homebrew は macOS 用のパッケージマネージャーです.
Homebrew で ROOT ビルドに必要な CMake をインストールします.\par
Homebrew を使うには, Xcode の Command Line Tools が必要になります.
ターミナルで次のコマンドを実行し, Xcode をインストールします.
\begin{lstlisting}
$ xcode-select --install
\end{lstlisting}
インストーラーの指示にしたがって, インストールを完了させます.
正常にインストールされていることを確認するため, 
\begin{lstlisting}
$ xcode-select -v
\end{lstlisting}
を実行します.バージョンが表示されていればインストール完了です.\par
次に, Homebrew をインストールします.
\url{https://brew.sh/index_ja.html}
上記のURLにあるコマンドをターミナルに入力します.
\begin{figure}[H]
    \centering
    \includegraphics[width=14cm]{Homebrew.png}
    \caption{HomebrewのHP画面.画面下のコマンドをコピーしてターミナル上で実行する.}
\end{figure}
\begin{lstlisting}[caption=Homebrewインストール時のスクリプト]
...
==> Installation successful!
==> Homebrew has enabled anonymous aggregate formulae and cask analytics.
Read the analytics documentation (and how to opt-out) here:
https://docs.brew.sh/Analytics
==> Homebrew is run entirely by unpaid volunteers. Please consider donating:
https://github.com/Homebrew/brew#donations
==> Next steps:
- Run `brew help` to get started
- Further documentation:
https://docs.brew.sh
\end{lstlisting}
が表示されれば, インストール完了です.\par

\subsubsection{CMakeのインストール}
Homebrew を用いて, CMakeをインストールします.
\begin{lstlisting}
$ brew install cmake
\end{lstlisting}
を実行すると, CMakeがインストールできます.

\subsubsection{ROOTのビルド}
\url{https://root.cern/install/build_from_source/#quick-start}に全て書いてあります.
ここでは例として, /opt/root をインストールディレクトリとした場合のビルド方法を示します.
\begin{lstlisting}
$ cd ~
$ mkdir root_build
$ git clone --branch latest-stable https://github.com/root-project/root.git root_src
$ cd root_build
$ sudo cmake -DCMAKE_INSTALL_PREFIX=/opt/root ../root_src
$ sudo cmake --build . --target install -j4
\end{lstlisting}
ビルドが成功したら, パスを通して終了です.
\$HOME/.zshrcに以下を追記し, シェルを再起動します.
\begin{lstlisting}
$ source /opt/root/bin/thisroot.sh
\end{lstlisting}
