\section{原理}

\subsection{光とは}
電子のエネルギー状態が, 高エネルギー状態から低エネルギー状態へ変化した時, このエネルギーの差分を原子の外に波動エネルギーとして放出する.この波動エネルギーを電磁波・光と呼ぶ.
位置$\bm{r}$, 時間tにおける電磁波は以下の式で表される.
\begin{itemize}
  \item 電場:$\bm{E} (\bm{r}, t) = \bm{E_0} \sin (\omega t - \bm{k} \cdot \bm{r} ) $
  \item 磁場:$\bm{B} (\bm{r}, t) = \bm{B_0} \sin (\omega t - \bm{k} \cdot \bm{r} ) $
\end{itemize}
$\bm{E_0}, \bm{B_0}$:定数,  $\bm{k}$:波数ベクトル,  $\omega$:角振動数とする.

\subsection{光の干渉}
光の波動性を見る1つの実験として, 干渉実験がある.以下, 光の干渉の原理について述べる.
\subsubsection{線状光源}
直線に並んだ振幅, 周波数が互いに等しいN個の光源を考える.各光源は等しい初期位相角をもっていると仮定する.ここで, j番目の光源が$r_j$離れた点で作る電場$E_j$は
\[
  E_j = E_0 \sin(\omega t - k r_j)
\]
と書ける.今, $E_0 \varpropto 1/{r_j}$なので, 定数$C_0$を用いて
\[
  E_j = \frac{C_0}{r_j} \sin(\omega t - k r_j)
\]



さらに, この光源を無限に並べた線状光源を考える.ここで, 考えている状況は, 各光源は非常に弱く, 光源の数Nは極めて多く, 光源間の間隔が無視できるほど小さい状況である.ここで, Dを線状光源全体の長さとすると, 線状光源の微小部分$\delta y_i$個の光源を含んでいる.(ここで, 光源はM個の微小部分に分けられているとする.$1 \leq i \leq M$)
\\
\begin{figure}[b]
  \begin{center}
    \includegraphics[width=6cm, bb = 0 0 600 200]{../SummerChallenge_lineray.png}
    \caption{線状光源}
  \end{center}
\end{figure}

\clearpage
このとき$\delta y_i$がPに作る電場は, 
\[
  E_i = \frac{C_0}{r_i} \sin(\omega t- k r_i ) \frac{\delta y_i N}{D}
\]
となる.ただし, $\delta y_i$は微小であり, この微小範囲内の各光源からPまで距離は一定であるとする.さらに一定値の$C_L$を
\[
  C_L = \frac{1}{D}  \lim_{N \to \infty} (C_0 N)
\]
と定義できる.M個の全部分によるPでの電場は, 
\[
  E = \sum_{i=1}^M \frac{C_L}{r_i} \sin(\omega t - k r_i) \delta y_i
\]
最後に, $\delta y_i$は無限小になるはずで, 
\[
  E = C_L \int_{-D/2}^{D/2} \frac{\sin(\omega t - k r)}{r} dy
\]
となる.ここで, $r = r(y)=\sqrt{R^2 \cos^2 \theta + (R\sin \theta - y )^2}$である.


\subsubsection{単スリット}
まず, 単スリットの場合どのように干渉が起こるかを確かめる.振幅, 周波数が互いに等しい長さDの線状光源を考える.線状光源の中心からyだけ離れた光源の微小部分dyが, 中心からxy平面上の角$\theta$の方向にRだけ離れた点Pに作る電場は, 
\[
  dE = \epsilon \frac{\sin(\omega t - kr)}{r} dy
\]
$r(y)$を$y$でテイラー展開すると, 
\begin{eqnarray*}
  r &=&r(0) + \frac{\partial r(0)}{\partial y} y + \frac{1}{2!} \frac{\partial^2 r(0)}{\partial y^2} y^2 + \cdots \\
  &=&R - y\sin\theta + \frac{y^2}{2R} \cos^2 \theta
\end{eqnarray*}
となる.いま, $R \gg y$なので, 第3項以降は無視できる.
\[
  dE = C_0 \frac{\sin(\omega t -k(R- y\sin \theta ))}{R- y \sin\theta} dy
\]
これの分母は$R \gg y$より$R$としてよいので, 
\[
  dE = C_0 \frac{\sin(\omega t -k(R- y\sin \theta ))}{R} dy
\]
これは, Rが十分大きいとき, $\theta$の全ての値に対して正しい.
\begin{eqnarray*}
  E &=& C_L \int_{-D/2}^{D/2} \frac{\sin(\omega t - k( R - y\sin\theta ))}{R} dy \\
  %&=& \frac{C_L}{R} \int_{-D/2}^{D/2} sin(\omega t - k( R - ysin\theta )) \\
  &=& \frac{C_L}{R(kD/2)\sin\theta} \sin\left(\frac{kD}{2} \sin \theta \right) \sin(\omega t -kR) \\
  &=& \frac{C_L D}{R\beta} \sin \beta \sin(\omega t - kR)
\end{eqnarray*}
$\beta = \frac{kD}{2} \sin\theta$とすると, 以上のようになり, 線状光源が作る電場が導かれた.このとき, 強度は$I(\theta) = \langle E^2\rangle_T$で求められるので, 
\begin{eqnarray*}
  I(\theta) &=& \langle E^2 \rangle_T\\
  &=& \langle \sin^2(\omega t - kR)\rangle_T \left(\frac{C_L D}{E}\right)^2 \left(\frac{\sin\beta}{\beta}\right)^2
\end{eqnarray*}
$\langle E^2\rangle_T$は$E^2$の十分に長い時間発展で, 周期 $2\pi / \omega$とすると, 
\begin{eqnarray*}
  \langle \sin^2(\omega t - kR)\rangle_T  &=& \frac{\omega}{2\pi} \int_{0}^{2\pi / \omega} \sin^2(\omega t - kR) dt \\
  %&=& \frac{\omega}{2\pi} \int_{0}^{2\pi / \omega} \frac{1}{2} (1-cos(2(\omega t -kR))) dt \\
  &=& \frac{1}{2}
\end{eqnarray*}
より, 
\[
  I(\theta) = \frac{1}{2} \left(\frac{C_L D}{R}\right)^2 \left(\frac{sin \beta}{\beta}\right)^2
\]
$I(0)$を求めると
\[
  I(0) = \frac{1}{2} \left(\frac{C_L D}{R}\right)^2
\]
なので, 
\[
  I(\theta) = I(\theta) \left(\frac{\sin \beta}{\beta}\right)^2
\]
と求まる.光の強度の変化と観測位置の関係は以下のように表される.\\

\begin{figure}[h]
  \begin{center}
    \includegraphics[width=10cm, bb = 0 0 1000 800]{SummerChallenge_slit.png}
    \caption{単スリットでの光の強度と角度の関係}
  \end{center}
\end{figure}

\clearpage

\subsubsection{2重スリット}
図のような, 幅$b$, 中心間隔$a$の2本の長いスリットがあるとする.スクリーン上のある点の光波に対する式を得るには, 2つの電場の和になるので, 
\begin{eqnarray*}
  E &=& \frac{C_L}{R} \int_{-b/2}^{b/2} F(z) dz + \frac{C_L}{R'} \int_{a-b/2}^{a+b/2} F(z) dz \\
  &\simeq& \frac{C_L}{R} \int_{-b/2}^{b/2} F(z) dz + \frac{C_L}{R} \int_{a-b/2}^{a+b/2} F(z) dz \\
\end{eqnarray*}
ここで, 
\begin{eqnarray*}
  F(z) &=& \sin(\omega t - k(R-z\sin\theta)) \\
  \alpha &=& \frac{ka}{2} \sin\theta \\
  \beta &=& \frac{kb}{2} \sin\theta
\end{eqnarray*}
これより, この式を簡単にすると, 
\[
  E = 2 \frac{C_L b}{R} \frac{\sin\beta}{\beta} \cos\alpha \sin(\omega t -kR + \alpha)
\]
であり, 単スリットの時と同様に強度$I(\theta) = \langle E^2\rangle_T$を求めると, $I(\theta) = \frac{1}{2} (\frac{C_L b}{R})^2$より
\[
  I(\theta) = 4I_0 \left(\frac{\sin \beta}{\beta}\right)^2 \cos^2 \alpha
\]
となる.\\
\\
\\
\\
\begin{figure}[h]
  \begin{center}
    \includegraphics[width=8cm, bb = 0 0 400 300]{../SummerChallenge_2slits.png}
    \caption{2重スリットでの光の強度と角度の関係}
  \end{center}
\end{figure}

\subsubsection{参考文献}
\begin{itemize}
  \item 「フーリエ光学(第3版)」森北博巳(森北出版株式会社)2012年初版
\end{itemize}
\clearpage


\subsection{光子を数える}
光量が極端に少なくなると, 光は光子として離散的になり, その数を数えることができるようになる.「光子数を数える」ということで, 光の粒子性を観測することができる.どのような機器で測定するかは, 次章にて述べる.
