
\subsection{データ解析用PCにログインする}
ここでは、自分のPCからLinuxにリモートでログインする方法を説明します。

\subsubsection{Windows Subsystem for Linux (WSL) を用いてLinuxを使う}
Windows10 以降で使用可能になった Windows Subsystem for Linux(WSL) は、 Windows10 から Linux の実行環境を実現するサブシステムです。簡単にいうと、 Windows のアプリケーションとして Linux を実行できます。

\vspace{1cm}
{\large \bf WSL の有効化}
\vspace{0.5cm}

まず、Windows 側で Linux Subsystem を有効化します。
\begin{enumerate}
  \item スタートボタンを右クリックし、「アプリと機能」をクリックします。
  \item 右上の「プログラムと機能」をクリックします。
  \item 左サイドバーの「Windows の機能の有効化または無効化」をクリックします。
  \item 「Windows Subsystem for Linux」を探して、チェックを入れます。
  \item 「OK」を押すと、インストールが始まります。インストール後、再起動します。
\end{enumerate}

\vspace{1cm}
{\large \bf Ubuntu のインストール}
\vspace{0.5cm}

\begin{enumerate}
  \item Microsoft Store で「WSL」と検索します。もしくは「Ubuntu」。
  \item アプリの中から「Ubuntu」を探し、インストールします。
  \item インストール作業が終わったら、スタートに追加される「Ubuntu」をを起動します。
  \item ユーザー名とパスワード を要求されるため、プロンプトに従って入力していきます。
\end{enumerate}
\begin{lstlisting}[caption=表示されるスクリプト例]
Installing, this may take a few minutes...
Please create a default UNIX user account. The username does not need to match your Windows username.
For more information visit: https://aka.ms/wslusers
Enter new UNIX username: [user name]
Enter new UNIX password: [password]
Retype new UNIX password: [password]
passwd: password updated succsessfully
Installation successful!
To run a command as administrator (user ``root''), use ``sudo''.
See ``man sudo_root'' for details.
\end{lstlisting}

\begin{enumerate}
  \setcounter{enumi}{4}
  \item パッケージのアップデート\\
        パッケージのアップデートは日々行ったほうが良いです。
\end{enumerate}
\begin{lstlisting}
sudo apt update
sudo apt upgrade
\end{lstlisting}

\vspace{1cm}
{\large \bf X window のインストール}
\vspace{0.5cm}

\begin{figure}[h]
  \begin{center}
    \includegraphics[width=14cm]{VcXsrv.png}
    \caption{VcXsrvのHP。Downloadをクリックします。}
  \end{center}
\end{figure}

Windows OS で X window (画像表示用・root用) を使用するため、「VcXsrv」をインストールします。
\begin{enumerate}
  \item \url{https://sourceforge.net/projects/vcxsrv/} にアクセスし、「Download」をクリックします。
  \item インストーラを起動し、オプションやインストール先などを指定してインストールします。
        \begin{itemize}
          \item Installation Option : 全てにチェックする。 (Full)
          \item Installation Folder : C:¥Program File¥VcXsrv
        \end{itemize}
  \item インストール作業が終わったら、スタートに追加される「VcXsrv」をを起動します。
  \item 起動するとウィザード画面が表示される。それぞれ選択し、設定を完了させます。
        \begin{itemize}
          \item Display settings : 好きなもので
          \item Client startup : 次へ (start no client)
          \item Extra settings : 次へ (Clipboard - Primary Selection, Native opengl
          \item Save configuration : 「Save configuration」をクリック
        \end{itemize}
  \item システムトレイにVcXsrvのアイコンが表示されていれば OK です。

  \item 追記:図が表示されなかったのでやったこと(WSL2だったから?) Ubuntu上で、xfce4をダウンロードします。dgm3 か lightdm かと聞かれるところで迷ったらdgm3を選んでください。
\begin{lstlisting}
sudo apt -y install xfce4-terminal xfce4
\end{lstlisting}
 \item エディタで ~/.bashrc をひらき、他の文字列とかぶらないところ(最後に書いとけば安全)に以下のことを記入して保存します。
\begin{lstlisting}
export DISPLAY=$(cat /etc/resolv.conf | grep nameserver | awk '{print $2}'):0.0
\end{lstlisting}
 \item 再度読み込めば、見えるようになるはず・・・。
\begin{lstlisting}
source ~/.bashrc
\end{lstlisting}
 \item [おまけ] GUIアプリで動作確認
\begin{lstlisting}    
sudo apt install x11-apps     サンプルアプリを入れます。
xeyes         これを入力すると、目玉が現れるはず。
\end{lstlisting}

参考URL(\url{https://root.cern.ch/install/build_from_source/})
\end{enumerate}


\begin{figure}[h]
  \begin{center}
    \includegraphics[width=14cm]{XQuartz.png}
    \caption{XQuartzのHP。dmgファイルをクリックします。}
  \end{center}
\end{figure}

macOS の場合は、「XQuartz」をインストールします。
\begin{enumerate}
  \item \url{https://www.xquartz.org/} にアクセスし、 dmg ファイルをダウンロードします。
  \item dmg ファイルを展開し、XQuartz.pkg をクリックして、起動します。
  \item インストーラに従って、XQuartz をインストールします。
  \item インストールが完了すると、/アプリケーション/ユーティリティ/ に XQuartz.app が作成されていますので、正常に起動するか確認します。
\end{enumerate}
=======

\subsection{データ解析用PCにログインする}
ここでは、自分のPCからLinuxにリモートでログインする方法を説明します。

\subsubsection{Windows Subsystem for Linux (WSL) を用いてLinuxを使う}
Windows10 以降で使用可能になった Windows Subsystem for Linux(WSL) は、 Windows10 から Linux の実行環境を実現するサブシステムです。簡単にいうと、 Windows のアプリケーションとして Linux を実行できます。

\vspace{1cm}
{\large \bf WSL の有効化}
\vspace{0.5cm}

まず、Windows 側で Linux Subsystem を有効化します。
\begin{enumerate}
  \item スタートボタンを右クリックし、「アプリと機能」をクリックします。
  \item 右上の「プログラムと機能」をクリックします。
  \item 左サイドバーの「Windows の機能の有効化または無効化」をクリックします。
  \item 「Windows Subsystem for Linux」を探して、チェックを入れます。
  \item 「OK」を押すと、インストールが始まります。インストール後、再起動します。
\end{enumerate}

\vspace{1cm}
{\large \bf Ubuntu のインストール}
\vspace{0.5cm}

\begin{enumerate}
  \item Microsoft Store で「WSL」と検索します。もしくは「Ubuntu」。
  \item アプリの中から「Ubuntu」を探し、インストールします。
  \item インストール作業が終わったら、スタートに追加される「Ubuntu」をを起動します。
  \item ユーザー名とパスワード を要求されるため、プロンプトに従って入力していきます。
\end{enumerate}
\begin{lstlisting}[caption=表示されるスクリプト例]
Installing, this may take a few minutes...
Please create a default UNIX user account. The username does not need to match your Windows username.
For more information visit: https://aka.ms/wslusers
Enter new UNIX username: [user name]
Enter new UNIX password: [password]
Retype new UNIX password: [password]
passwd: password updated succsessfully
Installation successful!
To run a command as administrator (user ``root''), use ``sudo''.
See ``man sudo_root'' for details.
\end{lstlisting}

\begin{enumerate}
  \setcounter{enumi}{4}
  \item パッケージのアップデート\\
        パッケージのアップデートは日々行ったほうが良いです。
\end{enumerate}
\begin{lstlisting}
sudo apt update
sudo apt upgrade
\end{lstlisting}

\vspace{1cm}
{\large \bf X window のインストール}
\vspace{0.5cm}

\begin{figure}[h]
  \begin{center}
    \includegraphics[width=14cm]{VcXsrv.png}
    \caption{VcXsrvのHP。Downloadをクリックします。}
  \end{center}
\end{figure}

Windows OS で X window (画像表示用・root用) を使用するため、「VcXsrv」をインストールします。
\begin{enumerate}
  \item \url{https://sourceforge.net/projects/vcxsrv/} にアクセスし、「Download」をクリックします。
  \item インストーラを起動し、オプションやインストール先などを指定してインストールします。
        \begin{itemize}
          \item Installation Option : 全てにチェックする。 (Full)
          \item Installation Folder : C:¥Program File¥VcXsrv
        \end{itemize}
  \item インストール作業が終わったら、スタートに追加される「VcXsrv」をを起動します。
  \item 起動するとウィザード画面が表示される。それぞれ選択し、設定を完了させます。
        \begin{itemize}
          \item Display settings : 好きなもので
          \item Client startup : 次へ (start no client)
          \item Extra settings : 次へ (Clipboard - Primary Selection, Native opengl
          \item Save configuration : 「Save configuration」をクリック
        \end{itemize}
  \item システムトレイにVcXsrvのアイコンが表示されていれば OK です。

  \item 追記:図が表示されなかったのでやったこと(WSL2だったから?) Ubuntu上で、xfce4をダウンロードします。dgm3 か lightdm かと聞かれるところで迷ったらdgm3を選んでください。
\begin{lstlisting}
sudo apt -y install xfce4-terminal xfce4
\end{lstlisting}
 \item エディタで ~/.bashrc をひらき、他の文字列とかぶらないところ(最後に書いとけば安全)に以下のことを記入して保存します。
\begin{lstlisting}
export DISPLAY=:0.0

それでもだめなら
export DISPLAY=$(cat /etc/resolv.conf | grep nameserver | awk '{print $2}'):0.0
\end{lstlisting}
 \item 再度読み込めば、見えるようになるはず・・・。
\begin{lstlisting}
source ~/.bashrc
\end{lstlisting}
 \item [おまけ] GUIアプリで動作確認
\begin{lstlisting}    
sudo apt install x11-apps     サンプルアプリを入れます。
xeyes         これを入力すると、目玉が現れるはず。
\end{lstlisting}

参考URL(\url{https://github.com/ymap-team/ROOT2021/wiki/Install_ROOT_Windows-Ubuntu-20.04/})
\end{enumerate}


\begin{figure}[h]
  \begin{center}
    \includegraphics[width=14cm]{XQuartz.png}
    \caption{XQuartzのHP。dmgファイルをクリックします。}
  \end{center}
\end{figure}

macOS の場合は、「XQuartz」をインストールします。
\begin{enumerate}
  \item \url{https://www.xquartz.org/} にアクセスし、 dmg ファイルをダウンロードします。
  \item dmg ファイルを展開し、XQuartz.pkg をクリックして、起動します。
  \item インストーラに従って、XQuartz をインストールします。
  \item インストールが完了すると、/アプリケーション/ユーティリティ/ に XQuartz.app が作成されていますので、正常に起動するか確認します。
\end{enumerate}
